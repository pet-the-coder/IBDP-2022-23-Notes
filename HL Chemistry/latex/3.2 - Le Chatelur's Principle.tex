\documentclass{article}
\usepackage[utf8]{inputenc}
\usepackage{caption}
\usepackage{amsmath}
\usepackage{amssymb}
\usepackage{mathtools}
\usepackage{multicol}
\usepackage{graphicx}
\usepackage{wrapfig}
\usepackage{float}
\usepackage[makeroom]{cancel}
\usepackage{mhchem}
\usepackage{pst-plot}

\graphicspath{ {../images/} }

\renewcommand{\baselinestretch}{1.5} % line spacing
\newcommand{\fline}{\par\noindent\rule{\textwidth}{0.1pt}} % horizontal line (wide)

\title{Topic 3 Equilibrium\\Lesson 2 Le Chatelur's Principle}
\author{Peter Zhang}

\begin{document}

\maketitle
\newpage
\tableofcontents
\newpage

% lesson 2
\section{Le Chatelur's Principle}
A system at equilbirium when subjected to a change will respond in such a way to minimize the effectt of the change.\\Whatever we do to the system, the system will respond \textbf{in the opposite way}\\Example: Add something to the system, \textbf{the system will react and try to remove it}\\Remove something, it will act to replace it.\\By the end, a new equilibrium is established.

\subsection{Point Form}
\begin{itemize}
\item \textbf{system at EQUILIBRIUM} when subjected to change $\rightarrow$ will minimize the effect of the change
\item add something in, the system will react to remove it
\item take something out, the system will find a way to replace it
\item \textbf{the system will ALWAYS head towards EQUILIBRIUM}
\end{itemize}


\pagebreak
\section{Types of Changes}
\begin{enumerate}
\item Concentration \begin{itemize} \item \textbf{$K_{c}$ should stay the same} \item the $k$ constant should remain constant \item \ce{^} in [] of reactants $\rightarrow$ shifts equilibrium to the right \item \ce{v} in [] of products shifts equilibrium to the right \end{itemize}
\item Pressure (focus on gases) \begin{itemize} \item \textbf{$K_{c}$ should stay the same} \item change in pressure/volume effects equilibrium depedant on the \# of gas moleculrs present\\ $$\ce{N2 + 3H2 \rightleftharpoons 2NH5}$$ \item There are 4 gas mol $\rightarrow$ 2 gas molecules \item look at overall reaction when it comes to pressure/volume changes \item (\ce{^ pressure -> v volume}) and vice verse \item If \ce{^ P, system responds by v P and shift towards side with smaller gas particles} \item if \ce{v P, shifts towards sides with more gas particles}\\ If equal \# of particles, \textbf{No change in equilibrium but change in rate} \end{itemize}
\item Temperature \begin{itemize} \item \textbf{$K_{c}$ is temp. dependant} \item \ce{^ in T, ^ in $K_{c}$ for \textbf{endothermic} -- overall system will be cooled to favor \textbf{products side}} \item \ce{^ in T, v in $K_{c}$ for \textbf{exothermic} -- overall system will reform reactants to decrease overall energy being added to system (in an already excess energy filled system)} \end{itemize}
\item Catalyst \begin{itemize} \item No change to $K_{c}$ or Equilibrium \end{itemize}

\end{enumerate}


\section{Examples}
\begin{enumerate}
\item 


\end{enumerate}






\end{document}