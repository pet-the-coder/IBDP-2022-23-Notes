\documentclass{article}
\usepackage[utf8]{inputenc}
\usepackage{caption}
\usepackage{amsmath}
\usepackage{amssymb}
\usepackage{mathtools}
\usepackage{multicol}
\usepackage{graphicx}
\usepackage{wrapfig}
\usepackage{float}
\usepackage[makeroom]{cancel}
\usepackage{mhchem}
\usepackage{pst-plot}

\graphicspath{ {../images/} }

\renewcommand{\baselinestretch}{1.5} % line spacing
\newcommand{\fline}{\par\noindent\rule{\textwidth}{0.1pt}} % horizontal line (wide)

\title{Topic 6 Equilibrium\\Lesson 3 Equilibrium Law}
\author{Peter Zhang}

\begin{document}

\maketitle
\tableofcontents
\newpage

% lesson 3
\section{Equilibrium Law}
\begin{itemize}
\item I - initial [] of reactants
\item C - change in [] (-) change for reactants (+) for products\\Losing reactants, gaining products | affected by coefficients
\item E - equilib []\ \ Initial $\pm$ change
\end{itemize}

\subsection{Example 1}

Determine $K_{c}$

\begin{align*}
\ce{SF2 + F2} &\rightleftharpoons \ce{SF6\ \ T=200^\circ C}\\
I: 0.52\ \ 0.4 & \\
C: \ \ \ \ & \\
E: 0.37\ \ \ & 
\end{align*}

If the equilbrium concentration of \ce{SF2} is 0.37, then the change can be calculated. The change in reactants can be found with the givens: \ce{SF2} is just ([] equil - [] initial) = 0.37 - 0.52 = -0.15. Th 

\begin{align*}
\ce{SF2 + F2} &\rightleftharpoons \ce{SF6\ \ T=200^\circ C}\\
I: 0.52\ \ 0.4 & \\
C: -0.15\ -0.30& +0.15\\
E: 0.37\ \ 0.1\ &|  0.15
\end{align*}

$$K_{c} = \ce{\frac{[SF6]}{[SF2][F2]^{2}}} = \frac{0.15}{0.37(0.1)^2} = 40.54$$


\subsection{Example 2}
\begin{align*}
\ce{CO + 2H2} &\rightleftharpoons \ce{CH3OH}\\
I: 0.42M\ 0.42M\ &| -
C: -x\ \ -2x&|\ +x\\
E: 0.42-x\ 0.42 * 2x &|\ x
\end{align*}

\ce{k_{c} = 5*10^{-4}}\ \ Find equilibrium []'s

Note if $K_{c}$ is less than $1*10^{-3}$ we can assume $\ce{reactant}_{[initial]} \& \ce{reactant}_{[equilib]}$. The only time you will end up with an $x^{3}$ situation is if the coefficient is 3?

Using the equation for $K_{c}$:
\begin{align*}
5*10^{4} &= \frac{x}{0.42(0.42)^{2}}\\
x &= 3.7*10^{-5}mol^{-3} smth idk \\
\ce{CH3OH]} &= 3.7x10^{-3}\\
\ce{[CO]} &=
collect from lesson slides &
\end{align*}



\section{Free Energy, $\triangle{G}$}
\begin{itemize}
\item If $\triangle{G}$ is negative - favor products
\item If $\triangle{G}$ is positive - favor reactants
\item If $\triangle{G}$ is zero - at equilibrium
\end{itemize}

$$\triangle{G} = -RT\ln{K_{c}}$$\ R = gas constant

Just another factor to decide which side you favor.







\end{document}