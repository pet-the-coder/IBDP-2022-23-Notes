\documentclass{article}
\usepackage[utf8]{inputenc}
\usepackage{caption}
\usepackage{amsmath}
\usepackage{amssymb}
\usepackage{mathtools}
\usepackage{multicol}
\usepackage{graphicx}
\usepackage{wrapfig}
\usepackage{float}
\usepackage[makeroom]{cancel}
\usepackage{mhchem}

\graphicspath{ {../images/} }

\renewcommand{\baselinestretch}{1.5} % line spacing
\newcommand{\fline}{\par\noindent\rule{\textwidth}{0.1pt}} % horizontal line (wide)

\title{Unit 2 Kinetics\\Supplementary Notes for Rate Equations}
\author{Peter Zhang}

\begin{document}

\maketitle
\newpage
\tableofcontents
\newpage

% textbook page 376
\section{Information Location}
This information is stored in the textbook on pages Topic 16 page 375.

\section{Rate Equations}
iIn topic 6 the idea of a rate equation as a the mathematical differential expression that expresses rate in terms of concentration. FOr example, consider the following reaction:

\begin{center}\ce{$x$A + $y$B -> $q$C + $p$D}\end{center}

where $x$, $y$, $q$, and $p$ are the \textbf{stoichiometry coefficients}. The \textbf{rate equation} is expressed as follows:

$$\ce{rate} = -\frac{1}{x}\frac{d\ce{[A]}}{dt} = -\frac{1}{y}\frac{d\ce{[B]}}{dt} = +\frac{1}{q}\frac{d\ce{[C]}}{dt} = +\frac{1}{p}\frac{d\ce{[D]}}{dt}$$

Therefore the rate of hte reaction depends on the \textbf{concentrations of the reactants}
\begin{itemize}
\item rate $\alpha$ [A]
\item rate $\alpha$ [B]
\item rate $\alpha$ [A][B]
\item rate = $k$[A][B]
\end{itemize}


This is the \textbf{rate equation} and general, it can be expressed in the form:

\begin{center}\ce{rate = $k$[A]^m[B]^n}\end{center}

\begin{itemize}
\item $k$ = rate constant
\item [A] = concentration of reactant A
\item [B] = concentration of reactant B
\item $m$ = exponendt in rate equation described as the \textbf{order with respect to reactant A}
\item $n$ = exponent in rate equation described as the \textbf{order with respect to reactant B}
\end{itemize}

The variables, $m$ and $n$ represent the \textbf{orders} with respect to each reactant, which convey how sensitive the rate of reaction is to changes in the concentrations of A and B. The \textbf{overall order of the reaction} is then defined as the sum of $m$ and $n$.

\begin{center}\ce{\textbf{overall reaction order} = $m$ + $n$} \end{center}

Rate equatinos can only be determined experimentally becuase the orders can only be deduced empirically.

\subsection{Example}
With the following example of:

$$\ce{NO2(g) + CO(g) -> NO(g) + CO2(g)}$$

The rate equation for the reaction of \ce{NO2(g)} with \ce{CO(g)} can be found experimentally to be:

$$\ce{rate} = k\ce{[NO2]}^{2}$$

Hence, the rate equation $\ce{rate} = k\ce{[A]^{m}[B]^{n}}$ corresponds to $\ce{rate} = k\ce{[NO2]}^{2}$. This means that $m = 2$ and $n = 0$

\begin{itemize}
\item the order with respect to \ce{NO2(g)} is equal to 2 (two)
\item the order with respect to \ce{CO(g)} is equal to 0 (zero)
\item the overall order with respect to the equations is 2 (two)
\item \textbf{NOTICE HOW THE OVERALL ORDER IS ALWAYS THE SAME}
\end{itemize}

One method of deducing the rate equations is to use the \textbf{method of inital rates}, the p[rinciple of which we introduced in topic 6. The value of the rate constant, $k$, is affected by temperature and its units are determined from the overall order of the reaction.

















\end{document}