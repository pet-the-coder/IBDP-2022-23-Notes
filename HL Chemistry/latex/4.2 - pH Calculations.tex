\documentclass{article}
\usepackage[utf8]{inputenc}
\usepackage{caption}
\usepackage{amsmath}
\usepackage{amssymb}
\usepackage{mathtools}
\usepackage{multicol}
\usepackage{graphicx}
\usepackage{wrapfig}
\usepackage{float}
\usepackage[makeroom]{cancel}
\usepackage{mhchem}
\usepackage{pst-plot}

\graphicspath{ {../images/} }

\renewcommand{\baselinestretch}{1.5} % line spacing
\newcommand{\fline}{\par\noindent\rule{\textwidth}{0.1pt}} % horizontal line (wide)

\title{Topic 8/18 Acids \& Bases\\Lesson 2 - pH Calculations}
\author{Peter Zhang}

\begin{document}

\maketitle
\tableofcontents
\newpage

% lesson 
\section{pH}
\textbf{pH} is just the $-\log $ of the concentration of \ce{H+} ions. \textbf{pOH} is the $-\log $ of the concentration of \ce{OH-} ions.

The pH scale is from 0-14 @298K and pH changes with temperature. When performing experiments in reactions with greater temperature change / enthalpy change, the pH will be affected.
\begin{itemize}
\item pH $<$ 7 = acidic
\item pH $>$ 7 = basic
\item pH = 7 = neutral
\end{itemize}
The pH value can be changed by \textbf{diluting} a solution x10 factor. \\Ex: 10mL of acid with 90mL of \ce{H2O}: such as \ce{PH3 -> ph = 4}.\\If you overdilute your solution, as long as you dilute the solution by an x10 factor, the pH will experience a great change. 

\section{Acid Base Neutralization}
\ce{HCl + NaOH $\rightleftharpoons$ NaCl + H2O}
\ce{H+(aq) + \cancel{\ce{Cl-(aq)}} + \cancel{\ce{Na+(aq)}} + OH-(aq) $\rightleftharpoons$ \cancel{\ce{Na+(aq)}} + \cancel{\ce{Cl-(aq)}} + H2O(l)}
\ce{H+(aq) + OH-(aq) $\rightleftharpoons$ H2O(l)}
\ce{H2O $\rightleftharpoons$ H+ + OH-}

Now we can find the equilibrium constant: $K_{c}$
\ce{K_{c} = \frac{\ce{[H+][OH-]}}{\ce{[H2O]}}}
Since only a small concentration of \ce{H2O} ionizes, we move it:
\ce{K_{c}\ce{[H2O]} = \ce{[H+][OH-]}}

Since $pKw = pH + pOH$ and $14 = pH + pOH$, we know that $pKw$ must be the same as $pH + pOH$, so $pKw = 14$

\ce{K_{w} = \ce{[H+][OH-]} = 1*10^{-14}} 
This is the same as:
\ce{pKw = $-\log{(\ce{[H+][OH-]})}$}

\subsection{IMPORTANT WHEN REPORTING VALUES}
\textbf{WE ONLY REPORT VALUES AS PH}, not pOH or other types. We only use pH value.


\subsection{Example}
\begin{enumerate}
    \item Find pH, [OH-] + pOH if [H+] = $3.7*10^{-9}moldm^{-3}$\begin{align*}
        1.0*10^{-14} &= \ce{[H+][OH-]}\\
        \ce{[OH-]} &= \frac{1*10^{-14}}{3.7*10^{-9}}\\
        &= 2.7*10^{-6}moldm^{-3}\\
        &\\
        pOH &= -\log{OH-}\\
        &= 5.57
    \end{align*}
\end{enumerate}

\subsection{Acid \& Base Strenghts}
Strong acids + bases fully dissociate / ionize\\
Example: \ce{HCl -> H+ + CL-} \\\ce{NaOH -> Na+ + OH-}

Weak acids + bases only partially dissociate / ionize
Exapmle: \ce{CH3COOH -> CH3COO- + H+} \\\ce{Na2CO3 -> 2Na+ + CO3^{-2}}

The closer a pH value is to zero, the stronge rthe acidic properties, however concentration is not equal to pH. \\
Example: pH of a lemon is 2 but we eat it and we don't die.




\end{document}