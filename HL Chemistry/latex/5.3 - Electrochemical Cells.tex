\documentclass{article}
\usepackage[utf8]{inputenc}
\usepackage{caption}
\usepackage{amsmath}
\usepackage{amssymb}
\usepackage{mathtools}
\usepackage{multicol}
\usepackage{graphicx}
\usepackage{wrapfig}
\usepackage{float}
\usepackage[makeroom]{cancel}
\usepackage{mhchem}
\usepackage{pst-plot}

\graphicspath{ {../images/} }

\renewcommand{\familydefault}{\sfdefault}
\renewcommand{\baselinestretch}{1.5} % line spacing
\newcommand{\fline}{\par\noindent\rule{\textwidth}{0.1pt}} % horizontal line (wide)

\title{Lesson Topic\\Lesson Number}
\author{Peter Zhang}

\begin{document}

\maketitle
\tableofcontents
\newpage

% lesson 
\section{Electrochemical Cells}
2 types
\begin{enumerate}
\item Voltaic cell (galvanic cell)\\generates electricity from spontaneous rxn's

ions on left and right + salt bridge (helps with movement of ions but do not affect the reaction at all) [spectator ions]

the goal of \textbf{salt bridge} only HELPS WITH MOVEMENT of charge.
$$\ce{Zn(s) \rightarrow Zn^{2+}_{(aq)}}$$

solution on left will allow for more solid zinc to become ions (ANODE) -- solution on right will thicken up the copper bar (CATHODE) -- \textbf{SALT BRIDGE} is just filled with cotton + \textbf{spectator ions} (potassium nitrate \ce{KNO3})

Net Reaction:
$$\ce{An(s) + Cu^{2+}_{(aq)} \rightarrow Zn^{2+}_{(aq)} + Cu(s)}$$

If you get a negative value when doing the math, the ANODE and CATHODE are REVERSED!!!\\
- or its just not spontaneous reaction $\rightarrow$ this means that its not a (galvanic cell) anymore\\
CATHOD and ANODE are just \textbf{electrodes}

IB may ask for diagram!!! \textbf{GET IMAGE FROM LESSON NOTES}

\subsection{Short Hand Form}
\subsubsection{Rules}
\begin{itemize}
\item Single vertical line represents a \textbf{phase boundary} i.e. solid electrode + aqueous salt
\item Double vertical line represents the \textbf{salt bridge} 
\item Aqeuous solid's of each electrode are placed next to the salt bridge
\item Anode goes on the left | Cathode goes on the right (electron flow from left to right)
\item Spectator ions are omitted
\item If half cell contains Z ions they are separated by a comma b/c they are in the same phase
\end{itemize}

Example: $$\ce{Zn(s) | Zn^{2+}_{(aq)} || Cu^{2+}_{(aq)} | Cu(s), Pt(s) = (Inert electrode)}$$
the oxidation anode $\rightarrow$ $e^{-} flow \rightarrow$ reduction cathode


Overall, to create a voltaiuc cell, any combination of anything can be used as long as the redox reactino provides a spontaneous overall reaction. The greater the distance between elements we are using, the higher the electric potential (Voltage) - see p24 data booklet.

\end{enumerate}

\section{Overall for Cells}
\begin{enumerate}
\item Higher on the reactivity series = more likely to \textbf{LOSE ELECTRONS}
\end{enumerate}

\section{Potential Difference, Voltage}
Measured elements against hydrogen
\begin{itemize}
\item H was used as a way to ensure a zero voltage reading \\Standard Hydrogen Electrode [SHE] = $E^{o}$ - standard electrode potential

Example:$$\ce{Zn(s) | Zn^{+2}(aq) || H+ (aq) | H2(g) | Pt(s)}$$

$E^{o} = -0.76V$\\All $E^{o}$ values refer to the reduction reaction and are generally dismissible

$E^{o}$ does not depend on total \# of \ce{e-} + not affected by coefficients in a blanaced rxn\\The more positive the more readily to reduce

\item 3 Ways of using $E^{o}$
\begin{enumerate}
\item Calculating all potential, $E^{o}_{cell}$ must be +ve for all spontaneous rxn
$$E^{o}_{cell} = E^{o}_{reduce} - E^{o}_{oxidation}$$
Example: $$\ce{Zn(s) | Zn^{2+}(aq) || Ag+ (aq) | Ag(s)}$$
\ce{Zn2+ + 2e- $\leftrightharpoons$ Zn\ \ $F^{o}= -0.76V$}\\
\ce{Ag+ + e- $\rightleftharpoons$ Ag\ \ $E^{o}= +0.80V$}\\
Therefore:
$$E^{o}_{cell} = E^{o}_{reduce} - E^{o}_{oxidation}$$
$$E^{o}_{cell} = 0.80V - (-0.76V)$$
$$E^{o}_{cell} = + 1.56V$$
\end{enumerate}

\item 2nd Way: the Sponaneity of an rxn
$$\triangle{G} = -nFE^{o}$$
where:\\n = \# of mols of \ce{e-} transferred\\F = Faraday constant = 96500 C/mol\\$E^{o}$ is just that

if $E^{o}$ is positive, GIBBS will be negative $\rightarrow$ spontaneous\\
if $E^{o}$ is negative, GIBBS will be positive $\rightarrow$ NOT spont\\
if $E^{o}$ is 0, GIBBS is 0 $\rightarrow$ rxn at equilibrium

\end{itemize}

\subsection{Example}
Reaction between Zn and Mg
\ce{Mg(s) | Mg2+ (aq) || Zn2+ (aq) | Zn(s)}
$$E^{o}_{cell} = E^{o}_{reduce} - E^{o}_{oxidation}$$
$$E^{o}_{cell} = -0.76 - (-2.37)$$
$$=1.61V$$

Finding the GIBBS

$$\triangle{G} = -nFE^{o}$$
$$=-2(96500)(1.61)$$
$$=-311kJ/mol$$ (usually given in J, not kJ)

\subsection{Example 2}
Comparing relativen oxidizing + reducing power of half cells
\begin{itemize}
\item A metal is able to reduce the ions of another metal taht has higher $E^{o}$\\Metals with low $E^{o}$ are strong reducing agents
\item A non-metal is able to oxidize the ions of another non-metal that has lower $E^{o}$\\Non-metals with high $E^{o}$ are strong oxidizing agents:\\
EXAMPLE: \ce{Cu2+ + 2e- \rightarrow Cu \ \ $E^{o} = +0.39V$}\\\ce{$\frac{1}{2}$Cl2 + e- $\rightarrow$ Cl-\ \ $E^{o} = +1.36V$}

Cl is the stronger oxidizing agent, \ce{Cu2+} would be the reducing agent
\end{itemize}

\section{Overview v2}
\begin{enumerate}
\item ALL $E^{o}$ value scan be found in data booklet... (plug and play into equations)
\end{enumerate}




\end{document}